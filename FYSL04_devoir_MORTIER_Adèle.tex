\documentclass[a4paper,10pt]{article}
\usepackage[utf8]{inputenc}
\usepackage[french]{babel}
\usepackage{frbib}
\usepackage{french}
\usepackage{graphicx}
\usepackage{geometry}
\geometry{
	a4paper,
	left=20mm,
	top=20mm,
}


\title{FYSL04 -- Analyse du discours de transmission de connaissance\\
	\textit{Comète ISON, que va-t-il se passer ?}\\ \vspace{0.3cm}
	\small Etude d'un article de vulgarisation scientifique}
\author{Adèle Mortier}

\begin{document}

\maketitle
\nocite{*}

\section{Introduction}
	Nous nous proposons dans ce rapport d'étudier le discours de transmission de connaissances à l'œuvre dans un article issu de \textit{Sciences et Vie}, magazine mensuel français de vulgarisation scientifique. L'article en question, intitulé \textit{Comète ISON, que va-t-il se passer ?} a été rédigé par Serge Brunier et publié le 16 novembre 2013. Le sujet principal de cet article est l'astrophysique, et plus particulièrement l'objet céleste C/2012 S1 (ISON), une comète rasante découverte en septembre 2012 et qui s'est désintégrée fin novembre 2013 au terme de son approche du Soleil \cite{ISON}.\\
	L'auteur, Serge Brunier est un reporter et écrivain français spécialisé dans la vulgarisation de l'astronomie auprès du public francophone. Il a écrit de nombreux ouvrages illustrés en rapport avec l'astronomie \cite{SB}. Ce journaliste se situe donc parfaitement dans le ``triangle de la vulgarisation'' à équidistance du scientifique, et du lecteur. Nous verrons dans la suite quels sont les procédés utilisés par Serge Brunier à des fins de vulgarisation et d'explication. Pour cela, on s'inspirera grandement du plan d'étude de Marie-Françoise Mortureux dans \cite{Mortureux1993}. On y ajoutera notamment des remarques concernant la spécificité du média, à savoir Internet.

\section{Repérage des paradigmes}
	Nous allons dans cette section passer en revue les outils formels permettant au journaliste d'introduire des définitions (\textit{paradigme définitionnel}), ou, plus souvent, des désignations (\textit{paradigme désignationnel}) se rapportant aux termes scientifiques décrits et/ou mentionnés. Ces termes techniques sont au nombre de cinq, avec un plus moins fort degré de familiarité pour le grand public. Le terme principal qui est aussi de sujet de l'article, est la comète ISON. Autour de ce terme ``gravitent'' quatre autres désignations : le Soleil, très familier, les satellites, bien connus également, les constellations, et enfin le ``nuage Oort''. Nous allons voir comment ces vocables sont introduits et explicités. 
	\subsection{Métalangage (paraphrase \textit{in praesentia})}
		\subsubsection{Syntagmes métalinguistiques}
			La manière la plus explicite de définir un terme nouveau est d'utiliser des syntagmes métalinguistiques, c'est-à-dire des noms ou des verbes faisant directement référence à la sémantique d'autres syntagmes ou parties du discours. Cela dit, on ne retrouve quasiment pas de traces de ces syntagmes au sein du texte à l'étude. Le seul qui pourrait être ici relevé est ``cette question'' (l. 9 de la version imprimable de l'article), qui fait référence à la phrase immédiatement en amont dans le texte, de nature interrogative (``Alors, « grande comète de 2013 » [...] ou « flop de l'année » ?'', l.7-9). Mais dans notre cas, le syntagme métalinguistique ne sert pas à définir un terme.\\
			Cette carence de syntagmes métalinguistiques s'explique par le fait que le discours s'adresse au grand public et tend par conséquent à perdre en densité formelle. On trouverait sans doute davantage de syntagmes métalinguistiques dans un manuel d'astrophysique, par exemple.
		\subsubsection{Paraphrases au métalangage ``estompé''}
			Dans cet article comme dans de nombreux textes de vulgarisation, on assiste en fait à un ``effacement du métalangage'', à un `` « dégradé » dans ses manifestations'' \cite{Mortureux1982}.\\
			On trouve ainsi dans l'article des qualifications moins strictes à l'aide de verbes comme \textit{être} ou \textit{marquer}, qui demeurent sémiotiquement ambigus.
			\begin{center}
				\footnotesize
				\begin{minipage}{0.7\textwidth}
					``Ce nuage, qui pourrait contenir des milliards de comètes, \textbf{serait} le vestige de la formation du système solaire et \textbf{marquerait} sa limite.'' (l.17-18)\\
					``ISON \textbf{pourrait donc être} l'une de ces comètes primordiales [...]'' (l.19)
				\end{minipage}
			\end{center}
			On voit que dans les contextes ci-dessus, le verbe \textit{être} et le verbe \textit{marquer} peuvent commuter avec le verbe \textit{désigner}, même si l'effet produit est un peu étrange. On note aussi la modalisation conditionnelle forte dans le premier mais surtout le deuxième exemple, signe que la définition tient aussi de la conjecture dans ce cas précis.\\
			Un procédé davantage usité dans notre texte est celui de la juxtaposition, alliée à la coordination : référent et co-référent apparaissent souvent de façon contiguë.
			\begin{center}
				\footnotesize
				\begin{minipage}{0.7\textwidth}
					``[...] les comètes\textbf{,} astres fantasques \textbf{et} fragiles'' (l.6-7)\\
					``Alors, « grande comète de 2013 »\textbf{,} comme je l'espérais ici voici quelques mois\textbf{,} « comète du siècle »\textbf{,} selon des confrères peut-être trop optimistes\textbf{, ou} « flop de l'année » ?'' (l.7-9)\\
					``Ce nuage, qui pourrait contenir des milliards de comètes, serait le vestige de la formation du système solaire \textbf{et} marquerait sa limite.'' (l.17-18)\\
					``elle n'échappera pas aux télescopes spatiaux en orbite autour de notre étoile\textbf{,} Stereo A, Stereo B\textbf{,} Solar Dynamics Observatory \textbf{et} Soho [...]'' (l.45-46)
				\end{minipage}
			\end{center}
			Dans tous les exemples sauf le troisième, la coordination apparaît pour articuler le dernier élément d'une juxtaposition de définitions. Ces définitions peuvent être mutuellement exclusives (deuxième exemple), ou additives (tous les autres exemples). La dernière énumération se démarque des autres en ce que les désignations juxtaposées sont en fait des instances du terme à définir\footnote{remarquons que le référent n'est pas forcément clair dès la première lecture dans le quatrième exemple, du fait de la longueur du syntagme nominal ayant pour noyau ``télescopes''}. Notons également que le deuxième exemple est marqué par le discours rapporté, et que par conséquent la source des différents ``discours'' est insérée entre chaque juxtaposition\footnote{ce n'est pas le cas pour la dernière des ``citations''. En revanche, tout porte à croire qu'il s'agit de l'appréciation potentielle du grand public.}. Cela donne au passage une dimension métadiscursive.\\
			Un dernier exemple issu de l'article se révèle en réalité trompeur:
			\begin{center}
				\footnotesize
				\begin{minipage}{0.7\textwidth}
					``[...] entre les constellations de la Couronne Boréale\textbf{,} le Bouvier\textbf{,} le Dragon \textbf{et} la Grande Ourse.'' (l.53-54)
				\end{minipage}
			\end{center}
			Tout porte à croire que la juxtaposition des noms de constellations (``Bouvier'', ``Dragon'', ``Grande Ourse'') constitue une définition de la ``Couronne Boréale'', vue comme la somme de ses composants (stellaires). Or, la Couronne Boréale est aussi une constellation, distincte des des trois autres. Avons nous affaire à une ambiguïté de formulation de la part du journaliste, qui aurait simplement voulu énumérer les différentes constellations ? Ou bien le journaliste cherchait-il effectivement à donner une définition (erronée) de la Couronne Boréale ? 
			
			
			
	\subsection{Diaphore}
	\subsection{Typographie}
	\subsection{Equivalence distributionnelle (paraphrase in absentia)}


\section{Analyse}
	\subsection{Relations lexicales}
	\subsection{Autres relations}
	\subsection{Distance sémantique}

\section{Interprétation}
	\subsection{Analyse du discours (théorie du discours)}
	\subsection{Lexique et vocabulaires (théorie du lexique)}






\begin{center}
	\footnotesize
	\begin{minipage}{0.7\textwidth}
		citation
	\end{minipage}
\end{center}
\medskip

\bibliographystyle{frcomplet}
\bibliography{bibliography}

\end{document}
