\documentclass[a4paper,10pt]{article}
\usepackage[utf8]{inputenc}
\usepackage[french]{babel}
\usepackage{frbib}
\usepackage{french}
\usepackage{hyperref}




\usepackage{graphicx}
\usepackage{geometry}
\geometry{
	a4paper,
	left=20mm,
	top=20mm,
}


\title{FYSL04 -- Analyse du discours de transmission de connaissance\\
	\textit{Comète ISON, que va-t-il se passer ?}\\ \vspace{0.3cm}
	\small Etude d'un article de vulgarisation scientifique}
\author{Adèle Mortier}

\begin{document}

\maketitle
\nocite{*}
\tableofcontents

\section*{Introduction} \label{intro}
	Nous nous proposons dans ce rapport d'étudier le discours de transmission de connaissances à l'œuvre dans un article issu de \textit{Sciences et Vie}, magazine mensuel français de vulgarisation scientifique. L'article en question, intitulé \textit{Comète ISON, que va-t-il se passer ?} a été rédigé par Serge Brunier et publié le 16 novembre 2013 \cite{Brunier2013b}. Le sujet principal de cet article est l'astrophysique, et plus particulièrement l'objet céleste C/2012 S1 (ISON), une comète rasante découverte en septembre 2012 et qui s'est désintégrée fin novembre 2013 au terme de son approche du Soleil \cite{ISON}.\\
	L'auteur, Serge Brunier est un reporter et écrivain français spécialisé dans la vulgarisation de l'astronomie auprès du public francophone. Il a écrit de nombreux ouvrages illustrés en rapport avec l'astronomie \cite{SB}. Ce journaliste se situe donc parfaitement dans le ``triangle de la vulgarisation'' à mi-chemin entre le scientifique et le lecteur. Nous verrons dans la suite quels sont les procédés utilisés par Serge Brunier à des fins de vulgarisation et d'explication. Pour cela, on s'inspirera grandement du plan d'étude de Marie-Françoise Mortureux dans \cite{Mortureux1993}. On y ajoutera notamment des remarques concernant la spécificité du média, à savoir Internet.

\section{Repérage des paradigmes} \label{reperage}
	Nous allons dans cette section passer en revue les outils formels permettant au journaliste d'introduire des définitions (\textit{paradigme définitionnel}), ou, plus souvent, des désignations (\textit{paradigme désignationnel}) se rapportant aux termes scientifiques décrits et/ou mentionnés. Ces termes techniques sont au nombre de cinq, avec un plus moins fort degré de familiarité pour le grand public. Le terme principal qui est aussi de sujet de l'article, est la comète ISON. Autour de ce terme ``gravitent'' quatre autres désignations : le Soleil, très familier, les télescopes (spatiaux), bien connus également, les constellations, et enfin le ``nuage de Oort''. Nous allons voir comment ces vocables sont introduits et explicités. 
	\subsection{Métalangage (paraphrase \textit{in praesentia})} \label{inpraesentia}
		\subsubsection{Syntagmes métalinguistiques}
			La manière la plus explicite de définir un terme nouveau est d'utiliser des syntagmes métalinguistiques, c'est-à-dire des noms ou des verbes faisant directement référence à la sémantique d'autres syntagmes ou parties du discours. Cela dit, on ne retrouve quasiment pas de traces de ces syntagmes au sein du texte à l'étude. Le seul qui pourrait être ici relevé est ``cette question'' (l. 9 de la version imprimable de l'article), qui fait référence à la phrase immédiatement en amont dans le texte, de nature interrogative (``Alors, « grande comète de 2013 » [...] ou « flop de l'année » ?'', l.7-9). Mais dans notre cas, le syntagme métalinguistique ne sert pas à définir un terme : il sert plutôt à désigner une alternative entre différentes définitions possibles.\\
			Cette carence de syntagmes métalinguistiques s'explique par le fait que le discours s'adresse au grand public et tend par conséquent à perdre en densité formelle. On trouverait sans doute davantage de syntagmes métalinguistiques dans un manuel d'astrophysique, par exemple.
		\subsubsection{Paraphrases au métalangage ``estompé''} \label{estompe}
			Dans cet article comme dans de nombreux textes de vulgarisation, on assiste en fait à un ``effacement du métalangage'', à un `` « dégradé » dans ses manifestations'' \cite{Mortureux1982}.\\
			On trouve ainsi dans l'article des qualifications moins strictes à l'aide de verbes comme \textit{être} ou \textit{marquer}, qui demeurent sémiotiquement ambigus.
			\begin{center}
				\footnotesize
				\begin{minipage}{0.7\textwidth}
					``Ce nuage, qui pourrait contenir des milliards de comètes, \textbf{serait} le vestige de la formation du système solaire et \textbf{marquerait} sa limite.'' (l.17-18)\\
					``ISON \textbf{pourrait donc être} l'une de ces comètes primordiales [...]'' (l.19)
				\end{minipage}
			\end{center}
			On voit que dans les contextes ci-dessus, le verbe \textit{être} et le verbe \textit{marquer} peuvent commuter avec le verbe \textit{désigner}, même si l'effet produit est un peu étrange. On note aussi la modalisation conditionnelle forte dans le premier mais surtout le deuxième exemple, signe que la définition tient aussi de la conjecture dans ce cas précis.\\
			
			Un procédé davantage usité dans notre texte est celui de la juxtaposition, alliée à la coordination : référent et co-référent apparaissent souvent de façon contiguë.\\
			Le premier exemple et le plus flagrant est celui du syntagme nominal ``la comète ISON'', qui apparaît à deux reprises dans l'article (l. 1 et l.12-13)\footnote{on rappelle qu'on a fixé comme vocable de base ``ISON'' et non ``la comète ISON'', pour des raisons qui seront justifiées plus tard.}. Dans ce syntagme, la co-référence apparaît en amont de la référence elle-même, sans même un signe de ponctuation pour les séparer (comme c'est normalement le cas dans une juxtaposition ``standard''). D'autres exemples peuvent aussi être relevés :
			\begin{center}
				\footnotesize
				\begin{minipage}{0.7\textwidth}
					``[...] les comètes\textbf{,} astres fantasques \textbf{et} fragiles'' (l.6-7)\\
					``Alors, « grande comète de 2013 »\textbf{,} comme je l'espérais ici voici quelques mois\textbf{,} « comète du siècle »\textbf{,} selon des confrères peut-être trop optimistes\textbf{, ou} « flop de l'année » ?'' (l.7-9)\\
					``Ce nuage, qui pourrait contenir des milliards de comètes, serait le vestige de la formation du système solaire \textbf{et} marquerait sa limite.'' (l.17-18)\\
					``elle n'échappera pas aux télescopes spatiaux en orbite autour de notre étoile\textbf{,} Stereo A, Stereo B\textbf{,} Solar Dynamics Observatory \textbf{et} Soho [...]'' (l.45-46)
				\end{minipage}
			\end{center}
			Dans tous les exemples sauf le troisième, la coordination apparaît pour articuler le dernier élément d'une juxtaposition de définitions. Ces définitions peuvent être mutuellement exclusives (deuxième exemple), ou additives (tous les autres exemples). La dernière énumération se démarque des autres en ce que les désignations juxtaposées sont en fait des instances du terme à définir\footnote{remarquons que le référent n'est pas forcément clair dès la première lecture dans le quatrième exemple, du fait de la longueur du syntagme nominal ayant pour noyau ``télescopes''}. Notons également que le deuxième exemple est marqué par le discours rapporté, et que par conséquent la source des différents ``discours'' est insérée entre chaque juxtaposition\footnote{ce n'est pas le cas pour la dernière des ``citations''. En revanche, tout porte à croire qu'il s'agit de l'appréciation potentielle du grand public.}. Cela donne au passage une dimension métadiscursive.\\
			Un dernier exemple issu de l'article se révèle en réalité trompeur:
			\begin{center}
				\footnotesize
				\begin{minipage}{0.7\textwidth}
					``[...] entre les constellations de la Couronne Boréale\textbf{,} le Bouvier\textbf{,} le Dragon \textbf{et} la Grande Ourse.'' (l.53-54)
				\end{minipage}
			\end{center}
			Tout porte à croire que la juxtaposition des noms de constellations (``Bouvier'', ``Dragon'', ``Grande Ourse'') constitue une définition de la ``Couronne Boréale'', vue comme la somme de ses composants stellaires. Or, la Couronne Boréale est aussi une constellation, qui s'avère donc distincte des trois autres. Dès lors, la question suivante se pose : avons-nous affaire à une ambiguïté de formulation de la part du journaliste, qui aurait simplement voulu énumérer les différentes constellations ? Ou bien le journaliste cherchait-il effectivement à donner une définition (erronée) de la Couronne Boréale ? Plus clairement, voulait-il dire :
			\begin{center}
				\footnotesize
				\begin{minipage}{0.7\textwidth}
					``[...] entre les constellations de la Couronne Boréale\textbf{,} \textit{c'est-a-dire} le Bouvier\textbf{,} le Dragon \textbf{et} la Grande Ourse.'' (l.53-54)
				\end{minipage}
			\end{center}
			ou bien :
			\begin{center}
				\footnotesize
				\begin{minipage}{0.7\textwidth}
					``[...] entre les constellations de la Couronne Boréale\textbf{,} \textit{du} Bouvier\textbf{,} \textit{du} Dragon \textbf{et} \textit{de la} Grande Ourse.'' (l.53-54)
				\end{minipage}
			\end{center}
			Nous somme ici confrontés à un cas particulier d'ambiguïté de portée, ou \textit{scope ambiguity}.
			
	\subsection{Diaphore} \label{diaphore}
		La ``fonction diaphorique [...] caractérise tout paratexte qui reprend sous forme condensée un fragment du texte'' \cite{Peraya1994}. Concrètement, une diaphore est soit une anaphore, soit une cataphore du léxème à définir ou d'un terme qui lui fait référence. Comme nous le verrons plus loin, cette notion est très liée à celle de thématisation.\\
		\subsubsection{Anaphores lexicales/grammaticales} \label{anaphore}
			Les anaphoriques les plus fréquents contiennent le terme ``comète'' qui comme on l'a vu est presque solidaire de l'acronyme ``ISON'':
			\begin{center}
				\footnotesize
				\begin{minipage}{0.7\textwidth}
					``la belle \textbf{comète}'' (l.4)\\
					``grande \textbf{comète} de 2013'' (l.7)
					``\textbf{comète} du siècle'' (l.8)\\
					``la \textbf{comète}'' (l.11, l.23, l.28, l.30, l.35, l.38, l.44, l.48)
				\end{minipage}
			\end{center}
			La dénomination de ``comète'' est de nature lexicale, car ce terme entrerait bien évidemment dans la définition de ISON \footnote{si une bien sûr une telle définition venait à exister dans un dictionnaire spécialisé. Nous utiliserons plus loin et en première approximation la définition donnée par Wikipédia}. Notons également que l'anaphore est décelable du fait de la présence d'un article défini singulier, ou à tout le moins du singulier. A ce titre, l'occurrence ``les comètes'' (l.6) ne donne qu'une définition indirecte de ISON et ne lui fait pas explicitement référence. Cela dit l'article défini singulier peut apparaître dans des cas d'anaphore plus ambigus, comme au dernier paragraphe, au sein d'un figement : ``ne tirons pas de plans sur la comète'' (l.54). L'expression tient ici du jeu de mots, et ``la comète'' désigne aussi bien la comète ``figée'' de l'expression  que la comète ISON. Outre le vocable de ``comète'', on retrouve d'autres lexèmes anaphoriques liés aux objets célestes :
			\begin{center}
				\footnotesize
				\begin{minipage}{0.7\textwidth}
					``l'\textbf{astre}'' (l.16, l.27) \\
					``Ce \textbf{nuage}'' (l.17) \\
					``notre \textbf{étoile}'' (l.41, l.43, l.46)
				\end{minipage}
			\end{center}
			On voit de nouveau que les articles (définis, démonstratifs, possessifs) jouent un rôle dans la mise en place de l'anaphore. On reste également dans le domaine de l'anaphore lexicale.\\
			On bascule dans l'anaphore grammaticale avec les mentions de :
			\begin{center}
				\footnotesize
				\begin{minipage}{0.7\textwidth}
					``\textbf{flop} de l'année'' (l.9) \\
					``jouer avec le \textbf{feu}'' (l.40) \\
				\end{minipage}
			\end{center}
			En effet, on voit mal le mot ``flop'' entrer dans la définition du mot ``comète'' dans un dictionnaire, et le Soleil -- auquel le mot ``feu'' fait référence selon les mêmes mécanismes que dans le figement précédent, ``tirer des plans sur la comète'' -- se trouve être composé en majeure partie de gaz... Nous sommes donc ici confrontés à des anaphores \textit{ad hoc}, qui ne sont résoluble qu'au vu du contexte bien particulier de l'article de presse.
		\subsubsection{Anaphores pronominales}
			Même si ces co-références jouissent d'une sémantique plus restreinte, elles sont au demeurant très présentes dans notre article :
			\begin{center}
				\footnotesize
				\begin{minipage}{0.7\textwidth}
					``elle'' (l.21, l.25, l.26, l.32, l.39, l.40, l.42, l.45, l.51)\\
					``l''' (l.33)\\
					``la'' (l.47)\\
					``son'' (l.5, l.20, l.21, l.23, l.32, l.50, l.54)\\
					``sa'' (l.39, l.50)
				\end{minipage}
			\end{center}
			Tous ces pronoms (personnels sujet ou complément, possessifs) font référence à ISON. Le genre féminin est employé dans la mesure ou le descripteur ``comète'', de genre féminin, est le descripteur ``naturel'' (le plus précis, le plus utilisé) de ISON. Remarquons également la profusion du pronom ``elle'' qui tend à personnifier la comète.
		\subsubsection{Cataphores} \label{cataphore}
			Les cataphores son bien moins fréquentes que les anaphores, dans la mesure où elles imposent une lecture non-linéaire et par conséquent plus attentive du texte, ce qui n'est pas l'effet recherché dans un article de vulgarisation. On trouve néanmoins quelques cataphores facilement compréhensibles :
			\begin{center}
				\footnotesize
				\begin{minipage}{0.7\textwidth}
					``la comète ISON'' (l.1, l.12-13) \\
					``le nuage Oort'' (l.16) \\
					``il est apparu que l'astre [...]'' (l.15-16)
				\end{minipage}
			\end{center}
			Les deux premiers exemples sont des cataphores immédiatement résolues, dans la mesure ou la définition (``comète'' ou ``nuage'') est située immédiatement en amont de sa référence. Le troisième exemple est formé par une formule impersonnelle antéposée à la mention de l'``astre''.\\
			Outre ces cataphores purement définitionnelles, on relève un certain nombre de cataphores circonstancielles, définissant l'état ou la situation de ISON :
			\begin{center}
				\footnotesize
				\begin{minipage}{0.7\textwidth}
					``Aujourd’hui, chauffée par le Soleil [...]'' (l.25)\\
					``En s'approchant du Soleil [...]'' (l.27-28)\\
					``Sous un très bon ciel, il est possible de l'apercevoir à l'œil nu ou aux jumelles [...]'' (l.33)\\
					``Là, durant quelques heures, chauffée à plus de 2000 degrés [...]'' (l.41-42)
				\end{minipage}
			\end{center}
			Ces structures s'appuient, on le voit, sur des mises en apposition à l'aide de participes adjectivés (exemples 1 et 4), de gérondifs (exemple 2), ou de formulations impersonnelles (exemple 3).
	\subsection{Typographie} \label{typo}
		Les mots-clefs du texte ou leur définition peuvent être mis en valeur grâce à des variations dans la typographie. Cette typographie prend, avec l'avènement de support Internet, une dimension nouvelle du fait notamment de l'utilisation de liens hypertextes, comme le souligne \cite{Toure2004} :
		\begin{center}
			\footnotesize
			\begin{minipage}{0.7\textwidth}
				``Le lien hypertexte par sa typographie, généralement une couleur et/ou
				un soulignement contribue à mettre en valeur certains éléments. Dans le cadre d’un texte de vulgarisation, les mots retenus sont ceux du spécialiste, ceux qui ont besoin d’être reformulés pour le grand public.''
			\end{minipage}
		\end{center}
		\subsubsection{Caractères}
			Les caractères typographiques sont sans doute les éléments les plus incontournables dans la mise en valeur du texte. On retrouve évidemment l'usage des virgules dans un but de juxtaposition des qualificatifs, comme il a été dit dans \ref{estompe}. On retrouve également les guillemets à des fins de citation :
			\begin{center}
				\footnotesize
				\begin{minipage}{0.7\textwidth}
					``Alors, \textbf{« grande comète de 2013 »}, comme je l'espérais ici voici quelques mois, \textbf{« comète du siècle »}, selon des confrères peut-être trop optimistes, ou \textbf{« flop de l'année »} ?'' (l.7-9)
				\end{minipage}
			\end{center}
			Le terme ``grande comète de 2013'' se retrouve effectivement dans un autre article de Serge Brunier daté de juin 2013 \cite{Brunier2013}. Le syntagme ``comète du siècle'' est effectivement très repris, sur les sites de BFM TV, Le Figaro, ou encore de l'Observatoire de Paris. Le syntagme ``flop de l'année'' en revanche, ne se retrouve \textit{a priori} nulle part. On peut penser qu'il s'agit d'une pseudo-citation que Serge Brunier attribue par défaut au grand public comme une idée reçue, ou bien que les guillemets permettent d'adoucir la fracture entre les registres de langue (avec l'usage du terme familier ``flop''). Les guillemets donnent à cet article de vulgarisation un aspect plurilogal et métadiscursif, dans une analyse qui rejoint celle de \cite{Toure2000}.\\
			On remarque aussi l'usage des parenthèses dans le but d'expliciter l'acronyme ISON, ``International Scientific Optical Network'' (l.14).\\
			Et, plus généralement, la ponctuation utilisée dans l'article -- de nombreux points de suspension, souvent en fin de paragraphe, des points d'interrogation qui créent un dynamisme de questions - réponses, des points d'exclamation pour mettre en valeur des détail étonnants sur la comète -- tout ceci tend à rendre la lecture plus conviviale et à éveiller la curiosité du lecteur.
		\subsubsection{Mise en forme}
			L'article reste plutôt sobre concernant la mise en forme du texte. On remarque que la première phrase de chaque paragraphe est mise en gras, mais cette contrainte de mise en forme est purement positionnelle, et demeure indépendante de la sémantique du texte. Et même, on pourrait dire que la mise en gras est négativement corrélée avec la présence d'un contenu informatif, car les premières phrases de paragraphes se révèlent être majoritairement des phrases d'accroche et non des définitions\footnote{sauf éventuellement ``ISON pourrait donc être l'une de ces comètes primordiales'', l.19}.\\
			Dernier détail de mise en forme : le lien hypertexte associé à ``système solaire'' (l.18). Ce lien apparaît en gras souligné sur la version imprimable et en gras rouge avec soulignage au survol dans la version en ligne. Le lien, qui malheureusement n'est plus actif, renvoyait à l'adresse \url{http://entre-terre-et-ciel.arte.tv/SystemeSolaire/Comprendre}, qui proposait sans doutes des contenus supplémentaires (vidéos...) à propos du mot-clef ``système solaire''.
	\subsection{Equivalence distributionnelle (paraphrase \textit{in absentia})}
		\cite{Mortureux1993} définit ainsi la paraphrase \textit{in absentia}:
		\begin{center}
			\footnotesize
			\begin{minipage}{0.7\textwidth}
				``Sa perception repose sur une analyse distributionnelle établissant une équivalence (formelle) entre des syntagmes qui, quelle que soit leur relation sémantique en langue, peuvent fonctionner dans le discours donné comme des coréférents''
			\end{minipage}
		\end{center}
		Nous n'avons pas d'exemple strict de ce genre de paraphrase dans notre article, car la paraphrase \textit{in absentia} se prête mieux aux documents longs (livres, manuels, traités...). Cela dit, on relève :
		\begin{center}
			\footnotesize
			\begin{minipage}{0.7\textwidth}
				``Pour eux, la comète ISON sera dans tous les cas une source d'information précieuse sur les \textbf{petits corps glacés qui orbitent loin du Soleil}.'' (l.13-14)
			\end{minipage}
		\end{center}
		Cette formulation est intéressante car elle laisse, à la manière d'une paraphrase \textit{in absentia}, une part importante à l'implicite. En effet, un lecteur novice qui ne connaîtrait pas la composition d'une comète pourrait penser que ``ISON'' et ``petits corps glacés qui orbitent loin du Soleil'' réfèrent à deux entités distinctes. Or, une comète, comme le souligne plus loin l'article (l.24), ``regorge de glaces, de gaz et de poussières''. ISON donne donc des informations précieuse sur elle-même, mais cela n'est pas évident en première lecture. Nous avons ici l'exemple d'une formulation relevant d'une lecture non-linéaire et d'un double discours \cite{Mortureux1984}.
		
\section{Analyse}
	\subsection{Relations lexicales}
		Dans \ref{diaphore}, nous avons pu relever les différents syntagmes qui servaient de co-références aux cinq termes scientifiques mentionnés dans notre article \footnote{``ISON'', ``Oort'', le ``Soleil'', les ``télescopes spatiaux'' et les ``constellations''}. Nous allons maintenant étudier es relations lexicales que ces co-références entretiennent avec leur terme de référence.
		\subsubsection{Synonymie}
			La relation de synonymie stricte ne se retrouve pas dans notre article, dans la mesure où les termes à définir sont principalement uniques en leur genre (``ISON'', ``Oort''...) et n'admettent par conséquent que des hyperonymes. On peut cependant relever la quasi équivalence entre ``Soleil'' et ``notre étoile'', qui repose essentiellement sur l'usage de l'article possessif. Notons également une équivalence entre deux co-références de ``ISON'' : ``comète'' et ``petits corps glacés qui orbitent loin du Soleil''. En effet la définition de ``comète'' est la suivante (\cite{comete}) :
			\begin{center}
				\footnotesize
				\begin{minipage}{0.7\textwidth}
					``Une comète est, en astronomie, un petit corps céleste constitué d'un noyau de glace et de poussière en orbite (sauf perturbation) autour d'une étoile''
				\end{minipage}
			\end{center}
		\subsubsection{Hyponymie/Hyperonymie}
			Cette relation lexicale entre classes est beaucoup plus fréquente car moins stricte. Nous nous proposons pour la définir de mener une classification relevant de l'analyticité, autrement dit, on se basera sur des définitions de dictionnaires ou encyclopédies.\\
			Pour ce qui est des co-références de ISON, on retrouve une hiérarchie à trois niveaux :
			\begin{itemize}
				\item \textbf{C/2012 S1 (ISON)} : ``comète rasante découverte en septembre 2012 et qui s'est désintégrée fin novembre 2013 au terme de son approche du Soleil'' \cite{ISON}. ISON est donc un hyponyme de ``comète rasante'', qui est trivialement un hyponyme de ``comète''. Autrement dit, \textbf{``comète'' est un hyperonyme de ``ISON''}.
				\item \textbf{Astre} : ``nom masculin, (latin \textit{astrum}, du grec \textit{astron}) : tout corps céleste naturel (Soleil, Lune, planète, comète, étoile, etc.)'' \cite{astre}. \textbf{``Astre'' est donc un hyperonyme de ``comète'', et, par transitivité, un hyperonyme de ``ISON''}.
			\end{itemize}
			Pour ce qui est des co-références des autres termes, on relève :
			\begin{itemize}
				\item \textbf{Soleil} : ``nom masculin
				(latin populaire \textit{soliculus}, du latin classique \textit{sol, solis}) : étoile autour de laquelle gravite la Terre. (Dans ce sens, s'écrit avec une majuscule.)'' \cite{soleil}. Donc \textbf{``étoile'' est un hyperonyme de ``Soleil''}\footnote{même si on a vu que l'adjonction d'un article possessif rendaient ``étoile'' et ``Soleil'' quasiment équivalents...}.
				\item \textbf{Oort} : ``le nuage d'Oort, aussi appelé le nuage d'Öpik-Oort, est un vaste ensemble sphérique hypothétique de corps'' \cite{oort}. On déduit de cette définition que \textbf{``nuage'' est un hyperonyme de ``Oort''}.
				\item De même, les différentes pages Wikipedia de Stereo A, Stereo B, Solar Dynamics Observatory et Soho font mention du lexème ``télescope''. \textbf{Donc tous ces noms propres sont des hyponymes de ``télescope spatial''}, au sens d'instances de classe.
			\end{itemize}
		\subsubsection{Métonymie et métaphore}
			Métonymie et métaphore sont des relations plus imagées que la synonymie et l'hyperonymie.\\
			La métonymie d'abord, qui consiste en bref à désigner le tout par la partie, se retrouve peu dans notre texte. On peut en trouver une forme atténuée au dernier paragraphe :
			\begin{center}
				\footnotesize
				\begin{minipage}{0.7\textwidth}
					``[...] nous devrions voir émerger sa \textbf{chevelure} [...] Puis, nuit après nuit, \textbf{elle} s'éloignera du Soleil et se montrera [...] mais ne tirons pas des plans sur la \textbf{comète} [...]'' (l.50-54)
				\end{minipage}
			\end{center}
			On note dans ce dernier paragraphe une ambiguïté latente entre la comète et sa chevelure. \footnote{cela est en fait scientifiquement pertinent, car ``la chevelure s'identifie fréquemment avec la tête de la comète, étant donné le faible diamètre relatif du noyau.'' \cite{comete}}. En effet, on peut se demander si le pronom ``elle'' rapporté dans l'exemple fait référence à la chevelure mentionnée immédiatement en amont, ou à la comète, évoquée immédiatement après. Or, on sait bien que c'est la comète toute entière qui progresse en direction du Soleil. Nous avons dont ici un cas de \textit{méronymie} (ou \textit{méréonymie}) impliquant un élément et l'un de ses constituants.
		
			On relève également dans notre article quelques métaphores, autrement dit, des substitutions imagées qui ne relèvent pas de l'analyticité :
			\begin{center}
				\footnotesize
				\begin{minipage}{0.7\textwidth}
					``Ce nuage [...] serait le \textbf{vestige} de la formation du système solaire et marquerait sa limite.'' (l.17-18) \\
					``Car ISON, dans sa trajectoire céleste, a choisi de jouer avec le \textbf{feu}...'' (l.40)
				\end{minipage}
			\end{center}
			Dans le premier exemple, ``Oort'' est assimilé à un vestige, or la définition de ``vestige'' est la suivante (\cite{vestige}) :
			\begin{center}
				\footnotesize
				\begin{minipage}{0.7\textwidth}
					\textbf{Vestige}, nom masculin, (latin \textit{vestigium}) : marque, trace laissée par quelque chose qui a été détruit : \textit{Les vestiges d'un ancien temple grec}.\\
					Littéraire. Ce qui reste du passé, d'un sentiment, d'une idée, etc. : \textit{Les vestiges d'une grandeur disparue}.
				\end{minipage}
			\end{center}
			On voit que cette définition n'a rien à voir avec l'astrophysique. Pour autant, Oort est bien une ``trace'' de la formation su système solaire.\\
			Quant au deuxième exemple, il tient à la fois de la métaphore et du jeu de mots. Comme il a été dit dans \ref{anaphore}, ``feu'' est une référence imagée au Soleil en même temps qu'il fait partie du figement ``jouer avec le feu''. En effet, la couleur est la texture du Soleil font penser qu'il s'agit d'une boule de feu, alors qu'il est fait essentiellement de gaz. De ce fait, ``feu'' est bien à classer parmi les métaphores et non parmi les métonymies.
	\subsection{Distance sémantique}
		Pour mesurer analytiquement la distance sémantique entre les différentes reformulations des termes définis dans notre article, on s'est basé sur la version française de l'ontologie WordNet \cite{WordNet1998}, à savoir WOLF, développée par Benoît Sagot \cite{Sagot2008} de l'équipe ALMAnaCH (INRIA / EPHE). Notre (courte) analyse a été écrite en Python à l'aide de l'API FreNetic disponible sur (\href{https://github.com/hardik-vala/FreNetic}{GitHub}).\\
		La notion de distance sémantique que nous introduisons ici repose sur les relations d'hyperonymie existant entre deux co-références. Dans WordNet, la notion d'hyperonymie est modélisée à l'aide de  \textit{synsets}, autrement dit d'ensembles de synonymes liés à un même concept cognitif. Un mot $M_1$ est hyperonyme d'un mot $M_2$ lorsque le \textit{synset} de $M_2$ est inclus dans celui de $M_1$. Sachant qu'un mot peut appartenir à plusieurs \textit{synsets}\footnote{par exemple, ``étoile'' peut appartenir à l'ensemble des objets célestes mais aussi à l'ensemble des danseurs}, il convient de choisir, pour chaque terme, le \textit{synset} qui correspond à l'interprétation qu'en fait notre article.\\
		Une fois les \textit{synsets} définis, la distance entre deux termes (et donc entre deux \textit{synsets}) est définie comme le nombre de \textit{synsets} à ``traverser'' pour passer du premier terme au second terme. Autrement dit, il s'agit de trouver la classe mère qui englobe les deux termes, et de sommer le nombre de niveaux intermédiaires à passer dans l'arbre de dépendances pour aboutir aux deux termes. Le tableau ci-après détaille les degrés de similarité entre les différentes co-références rencontrées dans notre article :
		\begin{center}
			\begin{tabular}{| c || c c c c c c c |}
				\hline
				& comète & corps & astre & soleil & feu & nuage & vestige\\
				\hline
				comète & 0 & 3 & 3 & - & - & - & -\\
				corps & 3 & 0 & 0 & - & - & - & -\\
				astre & 3 & 0 & 0 & - & - & - & -\\
				soleil & - & - & - & 0 & 8 & - & -\\
				feu & - & - & - & 8 & 0 & - & -\\
				nuage & - & - & - & - & - & 0 & 11\\
				vestige & - & - & - & - & - & 11 & 0\\
				\hline
			\end{tabular}
		\end{center}
		Comme on aurait pu s'y attendre, les termes utilisés dans le cadre d'une métaphore sont à une distance plus grande du terme de référence que ne le seraient des termes utilisés comme hyperonymes. Il aurait été intéressant de vérifier la distance entre ``comète'' et ``flop'' mais malheureusement ce dernier terme n'était pas répertorié dans l'ontologie.\\
		On a également gardé la trace de la classe ``englobante'' pour chaque paire de co-références :
		\begin{center}
			\begin{tabular}{| c | c |}
				\hline
				comète / corps & ``an object occurring naturally; not made by man'' \\
				comète / astre & ``an object occurring naturally; not made by man'' \\
				corps / astre & ``natural objects visible in the sky'' \\
				soleil / feu & ``an assemblage of parts that is regarded as a single entity'' \\
				nuage / vestige & ``an entity that has physical existence''\\
				\hline
			\end{tabular}
		\end{center}
		Comme on a vu que ``corps'' (céleste) et ``astre'' appartenaient au même \textit{synset}, on trouve la même classe englobante pour les couples ``comète / corps'' et ``comète / astre''. On note aussi que plus la relation entretenue est imagée, plus la classe englobante est vague : l'idée est qu'il a fallu remonter bien plus haut dans l'arbre de dépendances pour trouver un point commun entre les deux termes.\\
		Le Notebook contenant le code ayant servi à cette courte analyse est disponible sur mon \href{https://github.com/AdeleMortier/semantic_distance}{GitHub}.
\section{Interprétation}
	\subsection{Analyse du discours (théorie du discours)}
		\subsubsection{Thématisation}
			Dans notre cas, qui est celui d'un court article de presse, un invariant référentiel est clairement décelable et coïncide avec le thème principal du discours : il s'agit bien évidemment de la comète ``ISON''. ``ISON'' ou ses co-référents apparaissent en position de sujet (réel) à $25$ reprises dans notre article :
				\begin{center}
					\footnotesize
					\begin{minipage}{0.7\textwidth}
						``La \textbf{comète}, distante de moins de cent millions de kilomètres du Soleil, et qui \textbf{fonce} vers lui à raison de cinq millions de kilomètres par jour, \textbf{projette} derrière elle une magnifique chevelure.\\
						``En effet, si la \textbf{comète}, en frôlant le Soleil, \textbf{sera} évidemment invisible depuis la Terre, en revanche, elle \textbf{n'échappera pas} aux télescopes spatiaux en orbite autour de notre étoile, Stereo A, Stereo B, Solar Dynamics Observatory et Soho, qui la surveillent 24 h sur 24.''\\
						``nous devrions \textbf{voir émerger} sa \textbf{chevelure} au dessus de l'horizon est le 3 ou le 4 décembre, à l'aube.''
					\end{minipage}
				\end{center}
			Il faut aussi y ajouter la thématisation de certaines propriétés de la comète, telles son ``voyage'' (l.), sa ``chevelure'' (l. et l.), son ``éclat'' (l.), et son ``regain d'activité'' (l.) :
			\begin{center}
				\footnotesize
				\begin{minipage}{0.7\textwidth}
						``En s'approchant du Soleil, la \textbf{chevelure} de la comète va \textbf{s’accroître} encore puis ISON va développer une queue longue de plusieurs dizaines de millions de kilomètres...''\\
						``Son \textbf{éclat}, entre le début et la mi novembre, \textbf{a augmenté} d'un facteur cent !'' 
				\end{minipage}
			\end{center}
			\textit{A contrario}, ``ISON'', ses co-référents ou ses attributs n'apparaissent en position de prédicat qu'à $6$ reprises, notamment dans :
			\begin{center}
				\footnotesize
				\begin{minipage}{0.7\textwidth}
					``La \textbf{comète ISON}, que les astronomes amateurs et professionnels \textbf{suivent} avec curiosité depuis plus d'un an''\\
					``En fait, cette question n'intéresse que les astronomes amateurs, et le grand public, qui peut espérer \textbf{contempler} à l’œil nu le \textbf{passage de la comète} dans le ciel de la Terre'' 
				\end{minipage}
			\end{center}
			En outre, on peut remarquer que l'article esquisse un hiérarchie des thèmes, en introduisant des descriptions (plus courtes) des concepts ou des entités dont ``ISON'' dépend directement : le nuage de Oort et les télescopes spatiaux notamment. Le nuage de Oort apparaît d'abord en position de prédicat avant d'être lui-même introduit comme thème et défini
			par deux propositions coordonnées :
			\begin{center}
				\footnotesize
				\begin{minipage}{0.7\textwidth}
					``Très vite, il est apparu que l'astre provenait probablement du nuage de Oort, situé à des centaines de milliards de kilomètres du Soleil. \textbf{Ce nuage}, qui \textbf{pourrait contenir} des milliards de comètes, \textbf{serait} le vestige de la formation du système solaire et marquerait sa limite.'' 
				\end{minipage}
			\end{center}
			Ce qui fait de Oort la deuxième entité la plus importante -- quoique bien en deçà de ISON -- dans la hiérarchie thématique de l'article.\\
			Les ``télescopes'' sont aussi très brièvement définis, apparaissant d'abord en position d'agent par rapport à la comète, puis en position sujet dans une relative :
			\begin{center}
				\footnotesize
				\begin{minipage}{0.7\textwidth}
					``En effet, si la comète, en frôlant le Soleil, sera évidemment invisible depuis la Terre, en revanche, elle n'échappera pas aux \textbf{télescopes spatiaux en orbite autour de notre étoile}, Stereo A, Stereo B, Solar Dynamics Observatory et Soho, qui la \textbf{surveillent} 24 h sur 24.'' 
				\end{minipage}
			\end{center}
			Ce qui fait de ces ``télescopes'' la troisième entité de l'article jouissant d'une (très courte) définition.\\
			La hiérarchie des thèmes semble donc particulièrement abrupte dans notre article, comme on pouvait s'y attendre compte tenu de sa brièveté. A noter également que l'article pose de nombreux termes comme des présupposés, notamment les astres ``de base'' comme le Soleil, Mars, Jupiter la Terre ou la Lune; ainsi qu'un certain nombre de constellations, qui n'apparaissent qu'en position de prédicat.
		\subsubsection{Conceptualisation}
			Rappelons les enjeux de la conceptualisation selon \cite{Mortureux1993} :
			\begin{center}
				\footnotesize
				\begin{minipage}{0.7\textwidth}
					``Ensuite, la simple observation du paradigme dans un corpus de discours invite à rechercher le jeu d'une constante et de variables ; cela implique le choix d'une dénomination de base [...] Ce choix prend en compte, par conséquent, l'ensemble des discours référant à un domaine ou à un thème donné, quelle que soit, pour chacun d'eux, sa caractérisation socio-culturelle : scientifique, vulgarisateur, didactique, voire publicitaire...'' 
				\end{minipage}
			\end{center}
			Nous allons voir comment le thème de notre article -- la comète ISON -- est abordé selon différents points de vue, et donne lieu à une description pluriconceptuelle.\\
			ISON apparaît tout d'abord comme un concept scientifique, le point de vue des ``experts'' étant régulièrement convoqué par Serge Brunier :
			\begin{center}
				\footnotesize
				\begin{minipage}{0.7\textwidth}
					``La comète ISON, que \textbf{les astronomes amateurs et professionnels} suivent avec curiosité depuis plus d'un an [...]''\\
					``Tous les \textbf{observatoires} du monde ont un télescope pointé vers la belle comète [...]'' \\
					``Mais pour les \textbf{scientifiques}, les choses ne se présentent pas ainsi. Pour eux, la comète ISON sera dans tous les cas une source d'information précieuse sur les petits corps glacés qui orbitent loin du Soleil.'' \\
					``Très vite, \textbf{il est apparu} que l'astre provenait probablement du nuage de Oort, situé à des centaines de milliards de kilomètres du Soleil [...] ''\\
					``D'après les mesures réalisées à l'\textbf{observatoire} de Pico Veleta [...] ce regain d'activité pourrait être lié à une fragmentation du noyau, [...].''
				\end{minipage}
			\end{center}
			La présence de l'``expert'' peut être signalée par la mention du statut des individus (exemples 1 et 3), par l'allusion au matériel expérimental (exemples 2 et 5), ou encore, de façon implicite (exemple 4). Cette présence est par ailleurs plus marquée au début de l'article, principalement au paragraphes 1 et 2. Dans le cas du point de vue ``expert'', l'accent est mis sur le vif intérêt que suscite la comète (exemples 1 et 2), bien que celle-ci ne soit pas perçue comme importante en tant que telle. En effet, ce sont davantage les concepts que ISON sous-tend \footnote{origines, composition, comportement des comètes en général et rôle dans la formation du système solaire} qui éveillent la curiosité du monde scientifique. ISON est par conséquent abordée comme une simple instance d'une classe plus vaste et plus primitive.\\
			
			Le point de vue du grand public semble tout à fait opposé -- du moins c'est ainsi que l'article le présente :
			\begin{center}
				\footnotesize
				\begin{minipage}{0.7\textwidth}
					 ``Alors, « grande comète de 2013 », comme je l'espérais ici voici quelques mois, « comète du siècle », selon des confrères peut-être trop optimistes, ou « flop de l'année » ? En fait, cette question n'intéresse que les \textbf{astronomes amateurs, et le grand public}, qui peut espérer contempler à l’œil nu le passage de la comète dans le ciel de la Terre.''\\
					 ``Sous un très bon ciel, \textbf{il est possible} de l'apercevoir à l’œil nu et aux jumelles, dans la constellation de la Vierge. Mais la Lune va bientôt éclairer le ciel nocturne et probablement éclipser la comète jusqu'à la fin du mois...\\''
					 ``[...] \textbf{nous} devrions voir émerger sa chevelure au dessus de l'horizon est le 3 ou le 4 décembre, à l'aube. Puis, nuit après nuit, elle s'éloignera du Soleil et se montrera, en fin de nuit, de plus en plus haut sur l'horizon [...]''
				\end{minipage}
			\end{center}
			L'auteur semble inclure dans le ``grand public'' les astronomes amateurs et les journalistes spécialisés, qui devraient plutôt appartenir à la catégorie des ``vulgarisateurs''. Ce ``grand public'' fait son apparition dès le premier paragraphe, mais n'arrive au centre de l'attention que dans la seconde moitié de l'article. Contrairement aux scientifiques, le tout-venant témoigne d'un intérêt fort pour la comète en elle-même, moins perçue comme une entité physique que comme un évènement à ne pas rater. L'accent est mis sur les caractéristiques visuelles de la comète  -- ``éclat'', ``chevelure'' --  et non sur ses caractéristiques physiques intrinsèques; par ailleurs, les modalités selon lesquelles la comète pourra être aperçue sont plusieurs fois mentionnées.
		\subsubsection{Fonctionnalité du paradigme}
			Nous allons voir dans cette sous-section comment le clivage conceptuel entre monde scientifique et grand public est redoublé au moyen d'outils rhétoriques et stylistiques.\\
			Tout d'abord, le paradigme associé à ISON rend compte de ce clivage au niveau lexical et sémantique. D'un côté, les co-références ``petits corps glacés qui orbitent loin du Soleil'' et ``astre'' relèvent du monde scientifique, et portent sur les attributs physico-chimiques de la comète. D'un autre côté, les co-références ou allusions plus générales comme ``spectacle'', ``« comète du siècle »'', ``« flop de l'année »'', ``chevelure'', ou ``éclat'' qui parlent au grand public relèvent de caractéristiques plus superficielles, et tendent à définir la comète comme un évènement, voir un ``buzz''. En particulier, le lexème ``flop'' pourrait très bien s'appliquer à un mauvais film ou un mauvais tube... et on parle aussi très bien de ``casse du siècle'' de ``procès du siècle'' ou même de ``match du siècle''. Le paradigme associé à ISON rend donc bien compte de l'ambivalence du scientifique évoqué dans un contexte médiatique et médiatisé. Seul la co-référence ``comète'' semble revêtir une connotation tout à fait neutre.\\
			
			Ensuite, la dualité du paradigme est sensible dans l'agencement des phrases et la stylistique de l'auteur. Lorsque le journaliste évoque la facette scientifique du paradigme associé à ISON, ses phrases et ses formulations tiennent davantage du manuel spécialisé : ponctuation assez plate, agencement sous la forme très classique sujet-verbe-prédicat, contenu essentiellement informatif, incluant des dénominations précises et des données chiffrées (paragraphe 2 et 3 principalement).\\
			Lorsque la facette plus ``médiatique'' de la comète est évoquée, on relève une ponctuation bien plus vive (aposiopèses, exclamations, questions rhétoriques...) et plus globalement des marques accrues d'oralité :
			\begin{center}
				\footnotesize
				\begin{minipage}{0.7\textwidth}
					``Aujourd'hui, chauffée par le Soleil qu'elle voit se rapprocher \textbf{dangereusement}, elle se sublime et éjecte des milliers de tonnes de matière à chaque seconde !''\\
					``Car ISON, dans sa trajectoire céleste, a choisi de \textbf{jouer avec le feu}...''\\
					``Mais \textbf{ne tirons pas de plans sur la comète} avant d'assister à son passage face aux feux du Soleil.''\\
				\end{minipage}
			\end{center}
			On remarque ainsi l'emploi d'un lexique plus affectif, une certaine personnification de l'objet céleste, et l'emploi plus fréquent de figements, ce qui introduit des traits d'humour. Le discours comme la comètes gagnent une dimension ``spectaculaire''.
	\subsection{Lexique et vocabulaires (théorie du lexique)}






\begin{center}
	\footnotesize
	\begin{minipage}{0.7\textwidth}
		citation
	\end{minipage}
\end{center}
\medskip


\bibliographystyle{frcomplet}
\bibliography{bibliography}

\end{document}
