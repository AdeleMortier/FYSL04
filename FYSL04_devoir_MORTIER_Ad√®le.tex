\documentclass[a4paper,10pt]{article}
\usepackage[utf8]{inputenc}
\usepackage[french]{babel}
\usepackage{frbib}
\usepackage{french}
\usepackage{hyperref}




\usepackage{graphicx}
\usepackage{geometry}
\geometry{
	a4paper,
	left=20mm,
	top=20mm,
}


\title{FYSL04 -- Analyse du discours de transmission de connaissance\\
	\textit{Comète ISON, que va-t-il se passer ?}\\ \vspace{0.3cm}
	\small Etude d'un article de vulgarisation scientifique}
\author{Adèle Mortier}

\begin{document}

\maketitle
\nocite{*}
\tableofcontents

\section*{Introduction} \label{intro}
	Nous nous proposons dans ce rapport d'étudier le discours de transmission de connaissances à l'œuvre dans un article issu de \textit{Sciences et Vie}, magazine mensuel français de vulgarisation scientifique. L'article en question, intitulé \textit{Comète ISON, que va-t-il se passer ?} a été rédigé par Serge Brunier et publié le 16 novembre 2013. Le sujet principal de cet article est l'astrophysique, et plus particulièrement l'objet céleste C/2012 S1 (ISON), une comète rasante découverte en septembre 2012 et qui s'est désintégrée fin novembre 2013 au terme de son approche du Soleil \cite{ISON}.\\
	L'auteur, Serge Brunier est un reporter et écrivain français spécialisé dans la vulgarisation de l'astronomie auprès du public francophone. Il a écrit de nombreux ouvrages illustrés en rapport avec l'astronomie \cite{SB}. Ce journaliste se situe donc parfaitement dans le ``triangle de la vulgarisation'' à équidistance du scientifique, et du lecteur. Nous verrons dans la suite quels sont les procédés utilisés par Serge Brunier à des fins de vulgarisation et d'explication. Pour cela, on s'inspirera grandement du plan d'étude de Marie-Françoise Mortureux dans \cite{Mortureux1993}. On y ajoutera notamment des remarques concernant la spécificité du média, à savoir Internet.

\section{Repérage des paradigmes} \label{reperage}
	Nous allons dans cette section passer en revue les outils formels permettant au journaliste d'introduire des définitions (\textit{paradigme définitionnel}), ou, plus souvent, des désignations (\textit{paradigme désignationnel}) se rapportant aux termes scientifiques décrits et/ou mentionnés. Ces termes techniques sont au nombre de cinq, avec un plus moins fort degré de familiarité pour le grand public. Le terme principal qui est aussi de sujet de l'article, est la comète ISON. Autour de ce terme ``gravitent'' quatre autres désignations : le Soleil, très familier, les satellites, bien connus également, les constellations, et enfin le ``nuage Oort''. Nous allons voir comment ces vocables sont introduits et explicités. 
	\subsection{Métalangage (paraphrase \textit{in praesentia})} \label{inpraesentia}
		\subsubsection{Syntagmes métalinguistiques}
			La manière la plus explicite de définir un terme nouveau est d'utiliser des syntagmes métalinguistiques, c'est-à-dire des noms ou des verbes faisant directement référence à la sémantique d'autres syntagmes ou parties du discours. Cela dit, on ne retrouve quasiment pas de traces de ces syntagmes au sein du texte à l'étude. Le seul qui pourrait être ici relevé est ``cette question'' (l. 9 de la version imprimable de l'article), qui fait référence à la phrase immédiatement en amont dans le texte, de nature interrogative (``Alors, « grande comète de 2013 » [...] ou « flop de l'année » ?'', l.7-9). Mais dans notre cas, le syntagme métalinguistique ne sert pas à définir un terme.\\
			Cette carence de syntagmes métalinguistiques s'explique par le fait que le discours s'adresse au grand public et tend par conséquent à perdre en densité formelle. On trouverait sans doute davantage de syntagmes métalinguistiques dans un manuel d'astrophysique, par exemple.
		\subsubsection{Paraphrases au métalangage ``estompé''} \label{estompe}
			Dans cet article comme dans de nombreux textes de vulgarisation, on assiste en fait à un ``effacement du métalangage'', à un `` « dégradé » dans ses manifestations'' \cite{Mortureux1982}.\\
			On trouve ainsi dans l'article des qualifications moins strictes à l'aide de verbes comme \textit{être} ou \textit{marquer}, qui demeurent sémiotiquement ambigus.
			\begin{center}
				\footnotesize
				\begin{minipage}{0.7\textwidth}
					``Ce nuage, qui pourrait contenir des milliards de comètes, \textbf{serait} le vestige de la formation du système solaire et \textbf{marquerait} sa limite.'' (l.17-18)\\
					``ISON \textbf{pourrait donc être} l'une de ces comètes primordiales [...]'' (l.19)
				\end{minipage}
			\end{center}
			On voit que dans les contextes ci-dessus, le verbe \textit{être} et le verbe \textit{marquer} peuvent commuter avec le verbe \textit{désigner}, même si l'effet produit est un peu étrange. On note aussi la modalisation conditionnelle forte dans le premier mais surtout le deuxième exemple, signe que la définition tient aussi de la conjecture dans ce cas précis.\\
			
			Un procédé davantage usité dans notre texte est celui de la juxtaposition, alliée à la coordination : référent et co-référent apparaissent souvent de façon contiguë.\\
			Le premier exemple et le plus flagrant est celui du syntagme nominal ``la comète ISON'', qui apparaît à deux reprises dans l'article (l. 1 et l.12-13)\footnote{on rappelle qu'on a fixé comme vocable de base ``ISON'' et non ``la comète ISON'', pour des raisons qui seront justifiées plus tard.}. Dans ce syntagme, la co-référence apparaît en amont de la référence elle-même, sans même un signe de ponctuation pour les séparer (comme c'est normalement le cas dans une juxtaposition ``standard'').
			\begin{center}
				\footnotesize
				\begin{minipage}{0.7\textwidth}
					``[...] les comètes\textbf{,} astres fantasques \textbf{et} fragiles'' (l.6-7)\\
					``Alors, « grande comète de 2013 »\textbf{,} comme je l'espérais ici voici quelques mois\textbf{,} « comète du siècle »\textbf{,} selon des confrères peut-être trop optimistes\textbf{, ou} « flop de l'année » ?'' (l.7-9)\\
					``Ce nuage, qui pourrait contenir des milliards de comètes, serait le vestige de la formation du système solaire \textbf{et} marquerait sa limite.'' (l.17-18)\\
					``elle n'échappera pas aux télescopes spatiaux en orbite autour de notre étoile\textbf{,} Stereo A, Stereo B\textbf{,} Solar Dynamics Observatory \textbf{et} Soho [...]'' (l.45-46)
				\end{minipage}
			\end{center}
			Dans tous les exemples sauf le troisième, la coordination apparaît pour articuler le dernier élément d'une juxtaposition de définitions. Ces définitions peuvent être mutuellement exclusives (deuxième exemple), ou additives (tous les autres exemples). La dernière énumération se démarque des autres en ce que les désignations juxtaposées sont en fait des instances du terme à définir\footnote{remarquons que le référent n'est pas forcément clair dès la première lecture dans le quatrième exemple, du fait de la longueur du syntagme nominal ayant pour noyau ``télescopes''}. Notons également que le deuxième exemple est marqué par le discours rapporté, et que par conséquent la source des différents ``discours'' est insérée entre chaque juxtaposition\footnote{ce n'est pas le cas pour la dernière des ``citations''. En revanche, tout porte à croire qu'il s'agit de l'appréciation potentielle du grand public.}. Cela donne au passage une dimension métadiscursive.\\
			Un dernier exemple issu de l'article se révèle en réalité trompeur:
			\begin{center}
				\footnotesize
				\begin{minipage}{0.7\textwidth}
					``[...] entre les constellations de la Couronne Boréale\textbf{,} le Bouvier\textbf{,} le Dragon \textbf{et} la Grande Ourse.'' (l.53-54)
				\end{minipage}
			\end{center}
			Tout porte à croire que la juxtaposition des noms de constellations (``Bouvier'', ``Dragon'', ``Grande Ourse'') constitue une définition de la ``Couronne Boréale'', vue comme la somme de ses composants (stellaires). Or, la Couronne Boréale est aussi une constellation, distincte des des trois autres. Avons nous affaire à une ambiguïté de formulation de la part du journaliste, qui aurait simplement voulu énumérer les différentes constellations ? Ou bien le journaliste cherchait-il effectivement à donner une définition (erronée) de la Couronne Boréale ? 
			
			
	\subsection{Diaphore} \label{diaphore}
		La ``fonction diaphorique [...] caractérise tout paratexte qui reprend sous forme condensée un fragment du texte'' \cite{Peraya1994}. Concrètement, une diaphore est soit une anaphore, soit une cataphore du léxème à définir ou d'un terme qui lui fait référence. Comme nous le verrons plus loin, cette notion est très liée à celle de thématisation.\\
		\subsubsection{Anaphores lexicales/grammaticales} \label{anaphore}
			Les anaphoriques les plus fréquents contiennent le terme ``comète'' qui comme on l'a vu est presque solidaire de l'acronyme ``ISON'':
			\begin{center}
				\footnotesize
				\begin{minipage}{0.7\textwidth}
					``la belle \textbf{comète}'' (l.4)\\
					``grande \textbf{comète} de 2013'' (l.7)
					``\textbf{comète} du siècle'' (l.8)\\
					``la \textbf{comète}'' (l.11, l.23, l.28, l.30, l.35, l.38, l.44, l.48)
				\end{minipage}
			\end{center}
			La dénomination de ``comète'' est de nature lexicale, car ce terme entrerait bien évidemment dans la définition de ISON \footnote{si une bien sûr une telle définition venait à exister dans un dictionnaire spécialisé}. Notons également que l'anaphore est décelable du fait de la présence d'un article défini singulier, ou à tout le moins du singulier. A ce titre, l'occurrence ``les comètes'' (l.6) ne donne qu'une définition indirecte de ISON et ne lui fait pas directement référence. Cela dit l'article défini singulier peut apparaître dans des cas d'anaphore plus ambigus, comme au dernier paragraphe, au sein d'un figement : ``ne tirons pas de plans sur la comète'' (l.54). L'expression tient ici du jeu de mots, et ``la comète'' désigne aussi bien la comète ``figée'' de l'expression  que la comète ISON. Outre le vocable de ``comète'', on retrouve d'autres lexèmes anaphoriques liés aux objets célestes :
			\begin{center}
				\footnotesize
				\begin{minipage}{0.7\textwidth}
					``l'\textbf{astre}'' (l.16, l.27) \\
					``Ce \textbf{nuage}'' (l.17) \\
					``notre \textbf{étoile}'' (l.41, l.43, l.46)
				\end{minipage}
			\end{center}
			On voit de nouveau que les articles (définis, démonstratifs, possessifs) jouent un rôle dans la mise en place de l'anaphore. On reste également dans le domaine de l'anaphore lexicale.\\
			On bascule dans l'anaphore grammaticale avec les mentions de :
			\begin{center}
				\footnotesize
				\begin{minipage}{0.7\textwidth}
					``\textbf{flop} de l'année'' (l.9) \\
					``jouer avec le \textbf{feu}'' (l.40) \\
				\end{minipage}
			\end{center}
			En effet, on voit mal le mot ``flop'' entrer dans la définition du mot ``comète'' dans un dictionnaire, et le Soleil -- auquel le mot ``feu'' fait référence selon les mêmes mécanismes que dans le figement précédent, ``tirer des plans sur la comète'' -- se trouve être composé en majeure partie de gaz... Nous sommes donc ici confrontés à des anaphores \textit{ad hoc}, qui ne sont réolvables qu'en regard du contexte bien particulier de l'article.
			A COMPLETER EVENTUELLEMENT JOUER AVEC LE FEU QUI EST SIMILAIRE A TIRER DES PANS SUR LA COMTE ETC
		\subsubsection{Cataphores} \label{cataphore}
			Les cataphores son bien moins fréquentes que les anaphores, dans la mesure où elles imposent une lecture non-linéaire et par conséquent plus attentive du texte, ce qui n'est pas l'effet recherché dans un article de vulgarisation. On trouve néanmoins quelques cataphores facilement compréhensibles :
			\begin{center}
				\footnotesize
				\begin{minipage}{0.7\textwidth}
					``la comète ISON'' (l.1, l.12-13) \\
					``le nuage Oort'' (l.16) \\
					``il est apparu que l'astre [...]'' (l.15-16)
				\end{minipage}
			\end{center}
			Les deux premiers exemples sont des cataphores immédiatement résolues, dans la mesure ou la définition (``comète'' ou ``nuage'') est située immédiatement en amont de sa référence. Le troisième exemple est formé par une formule impersonnelle antéposée à la mention de l'``astre''.
			
			PRLER EVENTUELLEMENTDES PRONOMS
			
			
		DIRE QUE LENUAGE OORT ET LA COMETE ISON SONT DES CAS ANALOGUES!!
	\subsection{Typographie} \label{typo}
		Les mots-clefs du texte ou leur définition peuvent être mis en valeur grâce à des variations dans la typographie. Cette typographie prend, avec l'avènement de support Internet, une dimension nouvelle du fait notamment de l'utilisation de liens hypertextes, comme le souligne \cite{Toure2004} :
		\begin{center}
			\footnotesize
			\begin{minipage}{0.7\textwidth}
				``Le lien hypertexte par sa typographie, généralement une couleur et/ou
				un soulignement contribue à mettre en valeur certains éléments. Dans le cadre d’un texte de vulgarisation, les mots retenus sont ceux du spécialiste, ceux qui ont besoin d’être reformulés pour le grand public.''
			\end{minipage}
		\end{center}
		\subsubsection{Caractères}
			Les caractères typographiques sont sans doute les éléments les plus incontournables dans la mise en valeur du texte. On retrouve évidemment l'usage des virgules dans un but de juxtaposition des qualificatifs, comme il a été dit dans \ref{estompe}. On retrouve également les guillemets à des fins de citation :
			\begin{center}
				\footnotesize
				\begin{minipage}{0.7\textwidth}
					``Alors, \textbf{« grande comète de 2013 »}, comme je l'espérais ici voici quelques mois, \textbf{« comète du siècle »}, selon des confrères peut-être trop optimistes, ou \textbf{« flop de l'année »} ?'' (l.7-9)
				\end{minipage}
			\end{center}
			Le terme ``grande comète de 2013'' se retrouve effectivement dans un autre article de Serge Brunier daté de juin 2013 \cite{Brunier2013}. Le syntagme ``comète du siècle'' est effectivement très repris, sur les sites de BFM TV, Le Figaro, ou encore de l'Observatoire de Paris. Le syntagme ``flop de l'année'' en revanche, ne se retrouve \textit{a priori} nulle part. On peut penser qu'il s'agit d'une pseudo-citation que Serge Brunier attribue par défaut au grand public comme une idée reçue, ou bien que les guillemets permettent d'adoucir la fracture entre les registres de langues (avec l'usage du terme familier ``flop''). Les guillemets donnent à cet article de vulgarisation un aspect plurilogal et métadiscursif, dans une analyse qui rejoint celle de \cite{Toure2000}.\\
			On remarque aussi l'usage des parenthèses dans le but d'expliciter l'acronyme ISON, ``International Scientific Optical Network'' (l.14).\\
			Et, plus généralement, la ponctuation utilisée dans l'article -- de nombreux points de suspension, fréquemment en fin de paragraphe, des points d'interrogation qui créent un dynamisme de questions-réponses, des points d'exclamation pour mettre en valeur des détail étonnants sur la comète -- tout ceci tend à rendre la lecture plus conviviale et à éveiller la curiosité du lecteur.
		\subsubsection{Mise en forme}
			L'article reste plutôt sobre concernant la mise en forme du texte. On remarque que la première phrase de chaque paragraphe est mise en gras, mais il s'ensuit que cette mise en forme demeure indépendante de la sémantique du texte. Et même, on pourrait dire que la mise en gras est négativement corrélée avec la présence d'un contenu informatif, car les premières phrases de paragraphes se révèlent être majoritairement des phrases d'accroche et non des définitions\footnote{sauf éventuellement ``ISON pourrait donc être l'une de ces comètes primordiales'', l.19}.\\
			Dernier détail de mise en forme : le lien hypertexte associé à ``système solaire'' (l.18). Ce lien apparaît en gras souligné sur la version imprimable et en gras rouge avec soulignage au survol dans la version en ligne. Le lien, qui malheureusement n'est plus actif, renvoyait à l'adresse \url{http://entre-terre-et-ciel.arte.tv/SystemeSolaire/Comprendre}, qui proposait sans doutes des contenus supplémentaires (vidéos...) à propos du mot-clef ``système solaire''.
	\subsection{Equivalence distributionnelle (paraphrase \textit{in absentia})}
		\cite{Mortureux1993} définit ainsi la paraphrase \textit{in absentia}:
		\begin{center}
			\footnotesize
			\begin{minipage}{0.7\textwidth}
				``Sa perception repose sur une analyse distributionnelle établissant une équivalence (formelle) entre des syntagmes qui, quelle que soit leur relation sémantique en langue, peuvent fonctionner dans le discours donné comme des coréférents''
			\end{minipage}
		\end{center}
		Nous n'avons pas d'exemple strict de ce genre de paraphrase dans notre article, car la paraphrase \textit{in absentia} se prête mieux aux documents longs (livres, manuels, traités...). Cela dit, on relève :
		\begin{center}
			\footnotesize
			\begin{minipage}{0.7\textwidth}
				``Pour eux, la comète ISON sera dans tous les cas une source d'information précieuse sur les \textbf{petits corps glacés qui orbitent loin du Soleil}.'' (l.13-14)
			\end{minipage}
		\end{center}
		Cette formulation est intéressante car elle laisse, à la manière d'une paraphrase \textit{in absentia}, une part importante à l'implicite. En effet, un lecteur novice qui ne connaîtrait pas la composition d'une comète pourrait penser que ``ISON'' et ``petits corps glacés qui orbitent loin du Soleil'' réfèrent à deux entités distinctes. Or, une comète, comme le souligne plus loin l'article (l.24), ``regorge de glaces, de gaz et de poussières''. ISON donne donc des informations précieuse sur elle-même, mais cela n'est pas évident en première lecture. Nous avons ici l'exemple d'une formulation relevant d'une lecture non-linéaire et d'un double discours \cite{Mortureux1984}.
		
\section{Analyse}
	\subsection{Relations lexicales}
		Dans \ref{diaphore}, nous avons pu relever les différents syntagmes qui servaient de co-références aux cinq termes scientifiques mentionnés dans notre article \footnote{``ISON'', ``Oort'', le ``Soleil'', les ``satellites'' et les ``constellations''}. Nous allons maintenant étudier es relations lexicales que es co-références entretiennent avec leur référence.
		\subsubsection{Synonymie}
			La relation de synonymie stricte ne se retrouve pas dans notre article, dans la mesure où les termes à définir sont principalement uniques en leur genre (``ISON'', ``Oort''...) et n'admettent par conséquent que des hyperonymes. On peut cependant relever la quasi équivalence entre``Soleil'' et ``notre étoile'', qui repose essentiellement sur l'usage de l'article possessif. Notons également une équivalence entre deux co-références de ``ISON'' : ``comète'' et ``petits corps glacés qui orbitent loin du Soleil''. En effet la définition de ``comète'' est la suivante (\cite{comete}) :
			\begin{center}
				\footnotesize
				\begin{minipage}{0.7\textwidth}
					``Une comète est, en astronomie, un petit corps céleste constitué d'un noyau de glace et de poussière en orbite (sauf perturbation) autour d'une étoile''
				\end{minipage}
			\end{center}
		\subsubsection{Hyponymie/Hyperonymie}
			Cette relation lexicale entre classes est beaucoup plus fréquente car moins stricte. Nous nous proposons pour la définir de mener une classification relevant de l'analyticité, autrement dit, on se basera sur des définitions de dictionnaires ou encyclopédies.\\
			Pour ce qui est des co-références de ISON, on retrouve une hiérarchie à trois niveaux :
			\begin{itemize}
				\item \textbf{C/2012 S1 (ISON)} : comète rasante découverte en septembre 2012 et qui s'est désintégrée fin novembre 2013 au terme de son approche du Soleil. Source : \cite{ISON}. ISON est donc un hyponyme de ``comète rasante'', qui est trivialement un hyponyme de ``comète''. Autrement dit, \textbf{``comète'' est un hyperonyme de ``ISON''}.
				\item \textbf{Astre}, nom masculin, (latin \textit{astrum}, du grec \textit{astron}) : tout corps céleste naturel (Soleil, Lune, planète, comète, étoile, etc.). Source : \cite{astre}. \textbf{``Astre'' est donc un hyperonyme de ``comète'', et, par transitivité, un hyperonyme de ``ISON''}.
			\end{itemize}
			Pour ce qui est des co-références des autres termes, on relève :
			\begin{itemize}
				\item \textbf{Soleil}, nom masculin
				(latin populaire \textit{soliculus}, du latin classique \textit{sol, solis}) : étoile autour de laquelle gravite la Terre. (Dans ce sens, s'écrit avec une majuscule.). Source : \cite{soleil}. Donc \textbf{``étoile'' est un hyperonyme de ``Soleil''}\footnote{même si on a vu que l'adjonction d'un article possessif rendaient ``étoile'' et ``Soleil'' quasiment équivalents...}.
				\item \textbf{Oort} : le nuage d'Oort, aussi appelé le nuage d'Öpik-Oort, est un vaste ensemble sphérique hypothétique de corps. Source : \cite{oort}. On déduit de cette définition que \textbf{``nuage'' est un hyperonyme de ``Oort''}.
				\item De même, les différentes pages Wikipedia de Stereo A, Stereo B, Solar Dynamics Observatory et Soho font mention du lexème ``satellite''. \textbf{Donc tous ces noms propres sont des hyponymes de ``satellite''}, au sens 'instances de classe.
			\end{itemize}
		\subsubsection{Métonymies}
	\subsection{Autres relations}
	\subsection{Distance sémantique}

\section{Interprétation}
	\subsection{Analyse du discours (théorie du discours)}
	\subsection{Lexique et vocabulaires (théorie du lexique)}






\begin{center}
	\footnotesize
	\begin{minipage}{0.7\textwidth}
		citation
	\end{minipage}
\end{center}
\medskip


\bibliographystyle{frcomplet}
\bibliography{bibliography}

\end{document}
